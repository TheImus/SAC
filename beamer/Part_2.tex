\section{Soft Actor-Critic im kontinuierlichen Raum}
%======SAC Grundprinzip===========
\subsection{SAC Grundprinzip}

\begin{frame}{Kontinuierlicher Aktionsraum}
\begin{itemize}
\item kontinuierliche Aktionsräume benötigen
\item[] $\Rightarrow$ Approximation für Q-Funktion
\item[] $\Rightarrow$ Approximation für Strategie \\[12pt]
\item Schritt von Tabellen zu DNNs
\item Optimierung mittels gradient descent
\end{itemize}
\end{frame}

\begin{frame}{Funktionen und deren Netzwerke}
\begin{itemize}
\item State Value Funktion:
\item[] \textcolor{blue}{$V_{\psi}(s_{t})$} \,\,\,\,\,		$\rightarrow$ Skalar als Ausgabe \\[6pt]
\item Q-Funktion:
\item[] \textcolor{red}{$Q_{\theta}(s_{t},a_{t})$}		$\rightarrow$ Skalar als Ausgabe \\[6pt]
\item Strategie:
\item[] \textcolor{green}{$\pi_{\phi}(s_{t}|a_{t})$} \,	$\rightarrow$ Mittelwert und Kovarianz als Ausgabe $\Rightarrow$ Gauss \\[12pt]
\end{itemize}
Mit Parametervektoren \textcolor{blue}{$\psi$}, \textcolor{red}{$\theta$} und \textcolor{green}{$\phi$}
\end{frame}

%======SAC Update Regeln==========
\subsection{SAC Update Regeln}

\begin{frame}{State Value Funktion}
\begin{itemize}
\item eigenes Netzwetk nicht notwendig, aber
\begin{itemize}
\item stabilisiert Training
\item macht simultanes Training aller Netzwerke möglich
\end{itemize}
\end{itemize}
\end{frame}

\begin{frame}{Optimierung State Value Funktion}

\end{frame}

\begin{frame}{Q-Funktion}

\end{frame}

\begin{frame}{Optimierung Q-Funktion}

\end{frame}

\begin{frame}{Optimierung der Strategie}

\end{frame}

%======SAC Algorithmus============
\subsection{SAC Algorithmus}

\begin{frame}{Algorithmus (1/2)}

\end{frame}

\begin{frame}{Algorithmus (2/2)}

\end{frame}






