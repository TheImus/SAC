\section{Ergebnisse}
\begin{frame}
    \frametitle{Inhaltsverzeichnis}
    \tableofcontents[currentsection]
\end{frame}

\begin{frame}{Ziel der Experimente}
        \begin{itemize}
            \item Stabilität und Sample Komplexität im Vergleich zu anderen Algorithmen
            \begin{itemize}
                \item Kontinuierliche Aufgaben
                \item Verschiedene Schwierigkeitgrade
            \end{itemize}  
            \item OpenAI gym und rllab
        \end{itemize}
\end{frame}

\begin{frame}{Vergleich zu anderen Algorithmen}
    \begin{itemize}
        \item SAC
        \begin{itemize}
            \item Durchschnittswert (mean action)
            \item Feste und variable Temperatur (Anpassung im neuen Paper)
        \end{itemize} 
        \item PPO, DDPG
        \begin{itemize}
            \item Kein Exploration noise
        \end{itemize}
        \item TD3
        \item SQL mit zwei Q Funktionen
        \begin{itemize}
            \item Evaluation mit Exploration noise        
        \end{itemize}
    \end{itemize}
\end{frame}

\begin{frame}{Vergleich zu anderen Algorithmen}
    \begin{itemize}     
        \item 5 Instanzen mit einer Evaluation alle 1000 Schritte
        \item Schattierter Verlauf zeigt min und max der fünf Durchläufe
    \end{itemize}
    
\end{frame}

\begin{frame}
    \frametitle{Ergebnisse}
    % \includegraphics[width=.3\textwidth, height=0.4\textheight]{figures/rllab/Humanoid-v1.PNG}\hfill
    \includegraphics[width=.3\textwidth, height=0.4\textheight]{figures/rllab/Humanoid-v2.PNG}\hfill
    \includegraphics[width=.3\textwidth, height=0.4\textheight]{figures/rllab/halcheeathPNG.PNG}

    \includegraphics[width=.3\textwidth, height=0.4\textheight]{figures/rllab/Hopper.PNG}\hfill
    \includegraphics[width=.3\textwidth, height=0.4\textheight]{figures/rllab/antv2.PNG}\hfill
    \includegraphics[width=.3\textwidth, height=0.4\textheight]{figures/rllab/walker.PNG}\hfill \\
    \cite{picturesrllab}
\end{frame}


\subsection{Vergleich mit anderen Algortihmen}
\note{
    DDPG = Deep Deterministic Policy Gradiend 
    TD3 =  Twin deep Deterministic gradiend
    PPO = Proximal Policy Optimization
}
\begin{frame}
    \frametitle{Ergebnisse}
    \includegraphics[width=.3\textwidth]{figures/humanoid-rllab.pdf}\hfill
    \includegraphics[width=.3\textwidth]{figures/humanoid-gym.pdf}\hfill
    \includegraphics[width=.3\textwidth]{figures/half-cheetah.pdf}

    \includegraphics[width=.3\textwidth]{figures/hopper.pdf}\hfill
    \includegraphics[width=.3\textwidth]{figures/ant.pdf}\hfill
    \includegraphics[width=.3\textwidth]{figures/walker.pdf}\hfill
    \cite{SAC19}
\end{frame}

\begin{frame}
	\frametitle{Deterministisch vs Stochastisch }
	\includegraphics[width=0.9\textwidth, height=0.9\textheight]{figures/seeds-humanoid-rllab.pdf}\hfill
	\cite{SAC18}
\end{frame}

\begin{frame}
	\frametitle{Deterministische Evaluation }
	\includegraphics[width=0.5\textwidth, height=0.5\textheight]{figures/evaluation-ant.pdf}\hfill
	\includegraphics[width=0.5\textwidth, height=0.5\textheight]{figures/evaluation-half-cheetah.pdf}\hfill
	\cite{SAC18}
\end{frame}

\begin{frame}
	\frametitle{Reward Scale}
	\includegraphics[width=0.9\textwidth, height=0.9\textheight]{figures/reward-scale-ant.pdf}\hfill
	\cite{SAC18}
\end{frame}

\begin{frame}
	\frametitle{Target Network Update}
	\includegraphics[width=0.9\textwidth, height=0.9\textheight]{figures/soft-target-ant.pdf}\hfill
	\cite{SAC18}
\end{frame}

%\begin{frame}
%    \frametitle{Ergebnisse}
%    \includegraphics[width=0.8\textwidth]{figures/seeds-humanoid.pdf}
%\end{frame}


\subsection{Zusammenfassung}
\begin{frame}{Zusammenfassung}
        \begin{itemize}
            \item Soft actor critic vorgestellt
            \begin{itemize}
                \item Off policy Algorithmus
                \item Entropiemaximierung verbessert Stabilität
                \item temperature Wert wird gelernt statt fest zu setzen
            \end{itemize}
                \item Experimente in verschiedenen Umgebungen
                \item Besser als state-of-the-art Algorithmen  
                \item Stabiler als state-of-the-art Algorithmen  
        \end{itemize}
\end{frame}

